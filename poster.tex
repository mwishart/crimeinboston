%%%%%%%%%%%%%%%%%%%%%%%%%%%%%%%%%%%%%%%%%
% a0poster Landscape Poster
% LaTeX Template
% Version 1.0 (22/06/13)
%
% The a0poster class was created by:
% Gerlinde Kettl and Matthias Weiser (tex@kettl.de)
% 
% This template has been downloaded from:
% http://www.LaTeXTemplates.com
%
% License:
% CC BY-NC-SA 3.0 (http://creativecommons.org/licenses/by-nc-sa/3.0/)
%
%%%%%%%%%%%%%%%%%%%%%%%%%%%%%%%%%%%%%%%%%

%----------------------------------------------------------------------------------------
%	PACKAGES AND OTHER DOCUMENT CONFIGURATIONS
%----------------------------------------------------------------------------------------

\documentclass[a0,landscape]{a0poster}



\usepackage{multicol} % This is so we can have multiple columns of text side-by-side
\columnsep=100pt % This is the amount of white space between the columns in the poster
\columnseprule=3pt % This is the thickness of the black line between the columns in the poster

\usepackage[svgnames]{xcolor} % Specify colors by their 'svgnames', for a full list of all colors available see here: http://www.latextemplates.com/svgnames-colors

\usepackage{times} % Use the times font
%\usepackage{palatino} % Uncomment to use the Palatino font

\usepackage{graphicx} % Required for including images
\graphicspath{{figures/}} % Location of the graphics files
\usepackage{booktabs} % Top and bottom rules for table
\usepackage[font=small,labelfont=bf]{caption} % Required for specifying captions to tables and figures
\usepackage{amsfonts, amsmath, amsthm, amssymb} % For math fonts, symbols and environments
\usepackage[active]{srcltx}
\usepackage{wrapfig} % Allows wrapping text around tables and figures
\usepackage{caption}
\usepackage{vwcol}
\usepackage{tcolorbox}

\usepackage{pgf}
\usepackage{pgfpages}

\pgfpagesdeclarelayout{boxed}
{
  \edef\pgfpageoptionborder{0pt}
}
{
  \pgfpagesphysicalpageoptions
  {%
    logical pages=1,%
  }
  \pgfpageslogicalpageoptions{1}
  {
    \color{Gold},  
    border code=\pgfsetlinewidth{8pt}\color{Gold}\pgfstroke,%
    border shrink=\pgfpageoptionborder,%
    resized width=.95\pgfphysicalwidth,%
    resized height=.95\pgfphysicalheight,%
    center=\pgfpoint{.5\pgfphysicalwidth}{.5\pgfphysicalheight},
  }%
}

\pgfpagesuselayout{boxed}
\allowdisplaybreaks[4]
\newtheorem{theorem}{Theorem}[section]
\newtheorem{acknowledgement}[theorem]{Acknowledgement}
\newtheorem{algorithm}[theorem]{Algorithm}
\newtheorem{axiom}[theorem]{Axiom}
\newtheorem{case}[theorem]{Case}
\newtheorem{claim}[theorem]{Claim}
\newtheorem{conclusion}[theorem]{Conclusion}
\newtheorem{Def}[theorem]{Definition}
\newtheorem{thm}[theorem]{Theorem}
\newtheorem{prop}[theorem]{Proposition}
\newtheorem{condition}[theorem]{Condition}
\newtheorem{conjecture}[theorem]{Conjecture}
\newtheorem{corollary}[theorem]{Corollary}
\newtheorem{hyp}{Hypothesis}
\newtheorem{criterion}[theorem]{Criterion}
\newtheorem{definition}[theorem]{Definition}
\newtheorem{example}[theorem]{Example}
\newtheorem{exercise}[theorem]{Exercise}
\newtheorem{lemma}[theorem]{Lemma}
\newtheorem{notation}[theorem]{Notation}
\newtheorem{problem}[theorem]{Problem}
\newtheorem{proposition}[theorem]{Proposition}
\newtheorem{remark}[theorem]{Remark}
\newtheorem{solution}[theorem]{Solution}
\newtheorem{summary}[theorem]{Summary}
\renewcommand{\theequation}{\arabic{section}.\arabic{equation}}
\let\Section=\section
\def\section{\setcounter{equation}{0}\Section}
\newcommand{\R}{\mathbb{R}}
\newcommand{\Hh}{\mathcal{H}}
\newcommand{\E}{\mathbb{E}}
\newcommand{\tE}{\Tilde{\E}}
\newcommand{\cE}{\mathcal{E}}
\renewcommand{\P}{\mathbb{P}}
\newcommand{\tP}{\Tilde{\mathbb{P}}}
\newcommand{\Q}{\mathbb{Q}}
\newcommand{\1}{\bold{1}}
\newcommand{\e}{\epsilon}
\renewcommand{\labelitemii}{\circ}
\def\theequation{\thesection.\arabic{equation}}
\def\RR{\mathbb{R}}
\def\NN{\mathbb{N}}
\def\EE{\mathbb{E}}
\def\Tr{{\rm Tr}}
\def\barh{B^H }
\def\proof{{\noindent \it Proof\quad  }}
\def\problem{{\bf Problem\ \  \ }}
\def\wt{\widetilde}
\def\bfa{{\bf a}}
\def\bfb{{\bf b}}
\def\bfc{{\bf c}}
\def\bfd{{\bf d}}
\def\bfe{{\bf
e}}
\def\bff{{\bf f}}
\def\bfg{{\bf g}}
\def\bfh{{\bf h}}
\def\bfi{{\bf
i}}
\def\bfj{{\bf j}}
\def\bfk{{\bf k}}
\def\bfl{{\bf l}}
\def\bfm{{\bf
m}}
\def\bfn{{\bf n}}
\def\bfo{{\bf o}}
\def\bfp{{\bf p}}
\def\bfq{{\bf
q}}
\def\bfr{{\bf r}}
\def\bfs{{\bf s}}
\def\bft{{\bf t}}
\def\bfu{{\bf
u}}
\def\bfv{{\bf v}}
\def\bfw{{\bf w}}
\def\bfx{{\bf x}}
\def\bfy{{\bf y}}
\def\bfz{{\bf z}}
\def\cA{{\cal A}}
\def\cB{{\cal B}}
\def\cD{{\cal D}}
\def\cC{{\cal C}}
\def\cE{{\cal E}}
\def\cF{{\cal F}}
\def\cG{{\cal G}}
\def\cH{{\cal H}}
\def\fin{{\hfill $\Box$}}
\def\iint{{\int_0^t\!\!\!\int_0^t }}
\def\de{{\delta}}
\def\la{{\lambda}}
\def\si{{\sigma}}
\def\bbE{{\bb E}}
\def\rF{{\cal F}}
\def\De{{\Delta}}
\def\et{{\eta}}
\def\cL{{\cal L}}
\def\Om{{\Omega}}
\def\al{{\alpha}}
\def\Al{{\Xi}}
\def\be{{\beta}}
\def\Ga{{\Gamma}}
\def\g{\gamma}
\def\de{{\delta}}
\def\De{{\Delta}}
\def\Exp{{\hbox{Exp}}}
\def\si{{\sigma}}
\def\Si{{\Sigma}}
\def\tr{{ \hbox{ Tr} }}
\def\esssup {{ \hbox{ ess\ sup} }}
\def\la{{\lambda}}
\def\La{{\Lambda}}
\def\vare{{\varepsilon}}
\def \eref#1{\hbox{(\ref{#1})}}
\def\bart{{\underline  t}}
\def\barf{{\underline f}}
\def\barg{{\underline  g}}
\def\barh{{\underline h}}
\def\th{{\theta}}
\def\Th{{\Theta}}
\def\Om{{\Omega}}
\def\om{{\omega}}
\def\cS{{\cal S}}



\begin{document}

%----------------------------------------------------------------------------------------
%	POSTER HEADER 
%----------------------------------------------------------------------------------------

% The header is divided into three boxes:
% The first is 55% wide and houses the title, subtitle, names and university/organization
% The second is 25% wide and houses contact information
% The third is 19% wide and houses a logo for your university/organization or a photo of you
% The widths of these boxes can be easily edited to accommodate your content as you see fit



 


\begin{minipage}[b]{1.5\linewidth}
\begin{vwcol}[widths={0.35,0.65},sep=.8cm, justify=flush,rule=0pt,indent=1em]
\Huge \color{Blue} \textbf{A Look into Crime in Boston} \color{Black}\\ % Title
%\Huge\textit{An Exploration of Complexity}\\[1cm] % Subtitle
\huge \color{Black}\textbf{Julie Osborne, Chris McCooey, Mary Wishart}    
\vfill\null
%\columnbreak


\end{vwcol}
      
%\hspace {5cm} \huge University of Connecticut\\ 
  %Author(s)
 %\\ % University/organization
\end{minipage}
%
%\begin{minipage}[b]{0.25\linewidth}
%\color{DarkSlateGray}\Large \textbf{Contact Information:}\\
%Department Name\\ % Address
%University Name\\
%123 Broadway, State, Country\\\\
%Phone: +1 (000) 111 1111\\ % Phone number
%Email: \texttt{john@LaTeXTemplates.com}\\ % Email address
%\end{minipage}
%

\begin{multicols}{3} %` This is how many columns your poster will be broken into, a poster with many figures may benefit from less columns whereas a text-heavy poster benefits from more

\vspace{0.05cm} % A bit of extra whitespace between the header and poster content

%----------------------------------------------------------------------------------------


%----------------------------------------------------------------------------------------
%	ABSTRACT
%----------------------------------------------------------------------------------------

%\color{Navy} % Navy color for the abstract
\section*{Abstract}


%\end{abstract}

%----------------------------------------------------------------------------------------
%	INTRODUCTION
%----------------------------------------------------------------------------------------

\color{SaddleBrown} % SaddleBrown color for the introduction




%----------------------------------------------------------------------------------------
%	OBJECTIVES
%----------------------------------------------------------------------------------------
\color{black}\section{Introduction}

\columnsep=5pt
\columnseprule=0pt
%\color{blue}
\section{Utility Functions}


%------------------------------------------------
\section{Setup to our model}


  
%------------------------------------------------

\section{Weak Anticipation}



\vspace{0.3cm}

\begin{tcolorbox}[colback=blue!15!white, colframe = blue!70!white]

\end{tcolorbox}

%\footnotetext{In our finite discrete sample space, by equivalent we simply mean that $\Q$ is equivalent if $\forall i \in \{1,2,..,M\}, \Q(\omega_i)>0.$}
\vspace{0.25cm}

\begin{tcolorbox}

\end{tcolorbox}

\section{Value of Weak Information in a Complete Market}


\subsection{Application of Weak Information to Utility Functions}





%%%%%%%%%%%%%%%%%%%%%%%%%%%%%%%%%%







%\begin{theorem}[Baudoin, Gordina, M. 2018]
%If the operator $L$ satisfies 
%\[
%\Gamma_{2}^{L}(f)\geq\rho\Gamma^{L}(f),
%\]
%then the operator $\mathcal{L}$ satisfies the following generalized
%curvature-dimension inequality for any $f\in C^{\infty}\left(\mathbb{R}^{k}\times\mathbb{R}^{k}\right)$,
%\begin{align*}
%\Gamma_{2}(f) & \geq\left(\rho-\frac{C_{\sigma}}{2}\right)\Gamma(f)-\frac{C_{\sigma}}{2}\Gamma^{Z}(f),\\
%\Gamma_{2}^{Z}(f) & \geq0.
%\end{align*}
%
%\end{theorem}
%
%
%
%
%
%\begin{theorem}[Baudoin, Gordina, M. 2018]
%Let $P_{t}$ be the heat semigroup associated to $\mathcal{L}$. If
%$C_{\sigma}>2\rho$ and the operator $L$ satisfies 
%\[
%\Gamma_{2}^{L}(f)\geq\rho\Gamma^{L}(f),
%\]
%then for any $f\in C_{0}^{\infty}\left(\mathbb{R}^{k}\times\mathbb{R}^{k}\right)$,
%$t\geq0$ and $x\in\mathbb{R}^{k}\times\mathbb{R}^{k}$,
%\begin{align*}
% & \Gamma\left(P_{t}f\right)(x)+\frac{C_{\sigma}}{C_{\sigma}-2\rho}\Gamma^{Z}\left(P_{t}f\right)(x)\\
% & \leq e^{\left(C_{\sigma}-2\rho\right)t}\left(P_{t}\left(\Gamma(f)\right)+\frac{C_{\sigma}}{C_{\sigma}-2\rho}P_{t}\left(\Gamma^{Z}\left(f\right)\right)\right)(x).
%\end{align*}
%
%\end{theorem}
%
%
%
%We have the following Poincar\'e type inequality. 
%
%\begin{corollary}
%If $C_{\sigma}>2\rho$ then for any $f\in C_{0}^{\infty}\left(\mathbb{R}^{k}\times\mathbb{R}^{k}\right)$
%and $t\geq0$,
%\begin{align*}
% & P_{t}\left(f^{2}\right)-\left(P_{t}f\right)^{2}\\
% & \leq2\frac{e^{\left(C_{\sigma}-2\rho\right)t}-1}{C_{\sigma}-2\rho}\left(P_{t}\left(\Gamma(f)\right)+\frac{C_{\sigma}}{C_{\sigma}-2\rho}P_{t}\left(\Gamma^{Z}\left(f\right)\right)\right).
%\end{align*}
%\end{corollary}
%
%
%\item The results are more general, but constants aren't as sharp as the
%coupling technique.  
%\item The coupling technique yields a family of inequalities for $q\geq1$. 
%\end{itemize}












%%%%%%%%%%%%%%%%%%%%%%%%%%%%%%%%%%%%%%%%%%%%%%%%%%%%%%%%%%%%%%%%%






%\begin{thebibliography}{99}
%
%
%\bibitem{BGM2} Fabrice Baudoin, Maria Gordina, Phanuel Mariano, \emph{Gradient bounds for Kolmogorov type diffusions.} (2018), arXiv:1803.01436
%
%
%\end{thebibliography}




\end{multicols}
\vspace{1.5 in}


\end{document}
